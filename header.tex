
% using UTF8 as encoding for all files
\usepackage[utf8]{inputenc}

% provides '\includegraphics'
\usepackage{graphicx}

% provides auto-convert eps to pdfs
% \usepackage{epstopdf}

% provides cells spanning multiple rows in tables
\usepackage{multirow}

% American math society, base + symbol package (mathbb and so forth..)
\usepackage{amsmath,amssymb}

% provides double-stroked numbers, for e.g. unit matrix symbol 1
\usepackage{bbm}

% provides support for automated unit typesetting (e.g. 200 GeV)
\usepackage{units}

% provides \toprule etc. for nicer tables
\usepackage{booktabs}

% provides side-figures with text flow around them
\usepackage{wrapfig}

% language support, last option is standard for document
\usepackage[ngerman,english]{babel}

% subimport support for using cascaded \input command (gnuplot epslatex)
\usepackage{import}

%\usepackage[normal]{caption}

% change the (figure,etc.) counter, used for introduction
\usepackage{chngcntr}


% -----------------------------------------------------------
% Set / select font
% -----------------------------------------------------------

% Set the font
\usepackage[T1]{fontenc}

% Latin modern, successor of computer modern with correct umlauts
\usepackage{lmodern}	    



% -----------------------------------------------------------
% Somewhat more complex hyperlink setup 
% -----------------------------------------------------------

% provides extended colors, used here only for links
\usepackage{xcolor}

% link setup as suggested HERE: 
% http://tex.stackexchange.com/questions/823/remove-ugly-borders-around-clickable-cross-references-and-hyperlinks
\usepackage[pdfa,pdfencoding=auto,pdfusetitle
]{hyperref}
\definecolor{dark-red}{rgb}{0.4,0.15,0.15}
\definecolor{dark-blue}{rgb}{0.15,0.15,0.4}
\definecolor{medium-blue}{rgb}{0,0,0.5}
\hypersetup{
    colorlinks, linkcolor={dark-red},
    citecolor={dark-blue}, urlcolor={medium-blue},
    linktocpage=true
}

%
\setcapindent{1.5em}

% -----------------------------------------------------------

\usepackage{listings}

\definecolor{mygreen}{rgb}{0,0.6,0}
\definecolor{mygray}{rgb}{0.5,0.5,0.5}
\definecolor{mymauve}{rgb}{0.58,0,0.82}


\lstset{ %
  basicstyle=
    {\ttfamily\footnotesize},     % the size of the fonts that are used for the code
%  breakatwhitespace=true,          % sets if automatic breaks should only happen at whitespace
%  breaklines=true,                 % sets automatic line breaking
  commentstyle=\color{mygreen},    % comment style
  frame=leftline,
  xleftmargin=0.3em,
  keepspaces=true,                 % keeps spaces in text, useful for keeping indentation of code (possibly needs columns=flexible)
  keywordstyle=\color{blue},       % keyword style
  language=C++,                    % the language of the code
  numbers=none,                    % where to put the line-numbers; possible values are (none, left, right)
  showspaces=false,                % show spaces everywhere adding particular underscores; it overrides 'showstringspaces'
  showstringspaces=true,           % underline spaces within strings only
  showtabs=false,                  % show tabs within strings adding particular underscores
  stringstyle=\color{mymauve},     % string literal style
  tabsize=4,                       % sets default tabsize to 2 spaces
%  title=\lstname,                   % show the filename of files included with \lstinputlisting; also try caption instead of title
  morekeywords=[8]{Spinor,PsiEntry,SU3,Gaugefield,Lattice,LSite,LSiteIter,CommunicationBase}
}

% shorthand to listings inline work
\newcommand\cpp[1]{\lstinline{#1}}

% provides the cref command, automaticly puts "figure, table" etc. in refs
\usepackage{cleveref}


% ============================================================================
% Set a thick line for overfull boxes
% ----------------------------------------------------------------------------

\overfullrule=10pt


% ============================================================================
% list of new commands and shorthands
% ----------------------------------------------------------------------------

% The unit one 
\newcommand{\unitone}{\mathbbm{1}}

% The trace symbol
\DeclareMathOperator{\trace}{Tr}

% The "of order" O
\DeclareMathOperator{\ofOrder}{O}

% point-spread and optical transfer functions
\DeclareMathOperator{\PSF}{PSF}
\DeclareMathOperator{\OTF}{OTF}


% A TODO marker, easily found in the document printout and source code
\newcommand{\TODO}[1]{\textcolor{red}{[TODO]}\textcolor{blue}{(#1)}}
% A comment marker
\newcommand{\cmnt}[1]{\textcolor{orange}{[?: #1]}}

% see http://tex.stackexchange.com/questions/77816/hyperref-not-jumping-to-the-appropriate-location
% sees to that the links point to the top of a figure, not its caption
% do this here so it does not influence the epslatex plots
% done by gnuplot
\usepackage[all]{hypcap}


